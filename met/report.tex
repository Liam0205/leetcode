\documentclass[UTF8, final]{ctexart}
\usepackage{ifdraft}
\usepackage{hyperref}
\hypersetup{hidelinks}
\usepackage{geometry}
\geometry{scale = 0.81}
\usepackage{paralist}
\usepackage{listings}
\usepackage{xcolor}
\usepackage{amsmath, amssymb}
\renewcommand{\lstlistingname}{代码片段}
\lstdefinestyle{lfonts}{
    basicstyle   = \footnotesize\upshape\ttfamily,
    stringstyle  = \color{purple},
    keywordstyle = \color{blue!60!black}\bfseries,
    commentstyle = \color{olive}\scshape,
}
\lstdefinestyle{lnumbers}{
    numbers     = left,
    numberstyle = \tiny,
    numbersep   = 1em,
    firstnumber = 1,
    stepnumber  = 1,
}
\lstdefinestyle{llayout}{
    breaklines       = true,
    tabsize          = 2,
    columns          = spacefixed,
}
\lstdefinestyle{lgeometry}{
    xleftmargin      = 20pt,
    xrightmargin     = 0pt,
    frame            = tb,
    framesep         = \fboxsep,
    framexleftmargin = 20pt,
}
\lstdefinestyle{lgeneral}{
    style = lfonts,
    style = lnumbers,
    style = llayout,
    style = lgeometry,
    mathescape = true,
    escapechar = \%,
}
\lstdefinestyle{lcpp}{
    language = {C++},
    style = {lgeneral},
    morekeywords = {constexpr, noexcept, nullptr, override, size_t, final, assert}
}
\lstdefinestyle{lpy}{
    language = {Python},
    style = {lgeneral},
}
\lstdefinestyle{lsh}{
    language = {sh},
    style = {lgeneral},
}
\lstset{inputpath=src}
\usepackage{amsmath, amsthm}
\newtheorem{question}{题目}
\newtheorem{solution}{解答}

\title{面试题集解}
\author{\url{https://github.com/Liam0205/leetcode}}
\date{\today}

\ifdraft{\renewcommand{\lstinputlisting}[2][]{\relax}}{}
\begin{document}
\maketitle
\begin{abstract}
    题集解中的题,是面试题的题;题集解中的集,集自各种渠道的搜集;题集解中的解,是以 Cpp 实现的一家之言,仅供参考。

    若你有意,欢迎为这份题集解\href{https://github.com/Liam0205/leetcode/compare}{添砖加瓦}\footnote{\url{https://github.com/Liam0205/leetcode/compare}}。

    最后,愿你面试顺利!
\end{abstract}
\tableofcontents

\section{脑筋急转弯类型}
\begin{question}
已知你有无穷多的水,现在有两个提桶,一个装满水是3公升,一个装满水是5公升,这两个桶形状都不一样,怎么得到4公升的水。
\end{question}
\begin{solution}
设容积为 3 公升的桶为 A,容积为 5 公升的桶为 B。则有步骤:
\begin{compactitem}
    \item B 桶装满水;
    \item B 桶向 A 桶倒水至 A 桶满;
    \item A 桶清空;
    \item B 桶向 A 桶倒光剩余水;
    \item B 桶装满水;
    \item B 桶向 A 桶倒水至 A 桶满;
    \item B 桶剩余 4 公升水。
\end{compactitem}
\end{solution}

\section{偏数学问题}
\begin{question}
在一条线段上均匀随机选择两个点,将线段分成三小段。求这三小段能够组成三角形的概率。
\end{question}
\begin{solution}
设:线段长度为 $1$,两个点到线段左端点的距离分别为 $x$ 和 $y$,不妨设 $x \leqslant y$。则根据「任意两边之和大于第三边」的条件,组成如下不等式组
\[
  \begin{cases}
    x + (y - x) > 1 - y \\
    x + (1 - y) > y - x \\
    (y - x) + (1 - y) > x
  \end{cases}
\]
化简得
\[
  \begin{cases}
    y > 0.5 \\
    y < x + 0.5 \\
    x < 0.5
  \end{cases}
\]

由上述条件围成三角形的面积知结果 $P = 0.25$。

\end{solution}

\section{体系结构、操作系统相关}
\begin{question}
为什么 C++ 是「字节对齐」的?
\end{question}
\begin{solution}
本质原因是 CPU 访存时是按字节寻址的。如果 C++ 不按字节对齐,那么每次访存后,还要加上 0 -- 7 bit 的偏移量。如此,按照 Intel 的设计,会降低内存系统的性能。
\end{solution}

\section{基本算法的设计和实现}
\begin{question}
用 C++ 实现二分查找。
\end{question}
\begin{solution}
关键点:
\begin{compactitem}
    \item 不用递归;
    \item 边界条件处理。
\end{compactitem}
\lstinputlisting[style = lcpp]{bsearch.cc}
\end{solution}

\begin{question}
用 C++ 实现二分查找,对有重复元素的情况,考虑返回首次出现或者末次出现的位置。
\end{question}
\begin{solution}
关键点:
\begin{compactitem}
    \item 策略处理。
\end{compactitem}
\lstinputlisting[style = lcpp]{bsearch_with_policy.cc}
\end{solution}

\begin{question}
实现归并排序。
\end{question}
\begin{solution}
关键点:
\begin{compactitem}
    \item \lstinline[style = lcpp]|std::inplace_merge| 要求双向迭代器;
    \item 递归调用、递归终止条件。
\end{compactitem}
\lstinputlisting[style = lcpp]{merge_sort.cc}
\end{solution}

\begin{question}
实现二叉树的四种遍历:前序遍历、中序遍历、后序遍历、层次遍历。
\end{question}
\begin{solution}
关键点:
\begin{compactitem}
    \item 后序遍历注意结点输出条件;
    \item 层次遍历注意双指针方法的使用。
\end{compactitem}
\lstinputlisting[style = lcpp]{traverse_bitree.cc}
\end{solution}

\begin{question}
mapreduce为什么设计成map和reduce两个阶段
\end{question}
\begin{solution}
关键点:
\begin{compactitem}
    \item 计算步骤之间的数据依赖。
\end{compactitem}
\end{solution}

\begin{question}
设计并实现一个 hashmap,以开放地址法解决 Hash 碰撞。
\end{question}
\begin{solution}
关键点:
\begin{compactitem}
    \item 开放地址法的正确实现;
    \item 开放地址法在负载率超过 $0.8$ 时,效率迅速下降,注意扩张哈希表容量。
    \item 接口设计。
\end{compactitem}
\lstinputlisting[style = lcpp, escapechar = @]{hash_map.cc}
\end{solution}

\begin{question}
实现快速排序算法。
\end{question}
\begin{solution}
关键点:
\begin{compactitem}
    \item 主元的选择;
    \item C++ STL 的 \lstinline[style = lcpp]|std::partition| 并不保证主元在 \lstinline[style = lcpp]|cut| 位置;
    \item 利用循环减少递归次数。
\end{compactitem}
\lstinputlisting[style = lcpp, escapechar = @]{quick_sort.cc}
\end{solution}

\begin{question}
在给定元素顺序规则的情况下,实现对一个序列求其字典序下一个置换。
\end{question}
\begin{solution}
关键点:
\begin{compactitem}
    \item 从后往前看,找到第一个 drop 的位置,左边记为 \lstinline[style = lcpp]|valley| 右边记为 \lstinline[style = lcpp]|peak|;
    \item 在 \lstinline[style = lcpp]|peak| 之后的位置中,找到第一个在给定顺序中不先于 \lstinline[style = lcpp]|valley| 的元素,记为 \lstinline[style = lcpp]|i|;
    \item 逆序数增加且只增加 1。
\end{compactitem}
\lstinputlisting[style = lcpp, escapechar = @]{next_permutation.cc}
\end{solution}

\section{基本算法的运用}
\begin{question}
将矩阵顺时针旋转 90 度。
\end{question}
\begin{solution}
关键点:
\begin{compactitem}
    \item 资源的申请和释放;
    \item 函数封装;
    \item \lstinline[style = lcpp]|output[w][height - 1 - h] = input[h][w];|。
\end{compactitem}
\lstinputlisting[style = lcpp]{rotate_matrix.cc}
\end{solution}

\begin{question}
有一组非降序排列的数组,求它们的交集。
\end{question}
\begin{solution}
关键点:
\begin{compactitem}
    \item 利用非降序的特性;
    \item 求交集,最终结果逐渐减少,删除占主要操作,使用 \lstinline[style = lcpp]|std::list<T>|。
\end{compactitem}
\lstinputlisting[style = lcpp]{join_sorted_vector.cc}
\end{solution}

\begin{question}
有一整型数据组成的数组,相邻两个数之差的绝对值总是 1。要求实现一尽可能快的算法,寻找元素第一次出现的位置。
\end{question}
\begin{solution}
关键点:
\begin{compactitem}
    \item 利用\emph{相邻两个数之差的绝对值总是 1}的条件,跳跃搜索。
\end{compactitem}
\lstinputlisting[style = lcpp]{search_with_jump.cc}
\end{solution}

\begin{question}
现有两个以非增序排列的数组,求两个数组中第 n 小的数。
\end{question}
\begin{solution}
关键点:
\begin{compactitem}
    \item 归并;
    \item 计数器。
\end{compactitem}
\lstinputlisting[style = lcpp]{find_topK_with_merge.cc}
\end{solution}

\begin{question}
\label{q:stocks}
有一非负实数序列表示某一股票的价格变化。现要求设计一算法,尽可能快地寻找一次买入和卖出条件下所能获利的最大值。
\end{question}
\begin{solution}
关键点:
\begin{compactitem}
    \item 算法复杂度下界是 $O(n)$;
    \item 两两做差,求得股票每日价格变化值;
    \item 问题转换为子序列最大和问题,动态规划可破。
\end{compactitem}
\lstinputlisting[style = lcpp]{find_max_profit.cc}
\end{solution}

\begin{question}
设计一个快速求幂方的算法,以计算 $6^{15}$。
\end{question}
\begin{solution}
关键点:
\begin{compactitem}
    \item 剔除幂的二进制表示的尾部 0;
    \item 非递归,避免重复计算。
\end{compactitem}
\lstinputlisting[style = lcpp]{fast_power.cc}
\end{solution}

\begin{question}
输入有序数组,输出哈夫曼树。
\end{question}
\begin{solution}
关键点:
\begin{compactitem}
    \item 哈弗曼树的定义;
    \item 链表的运用。
\end{compactitem}
\lstinputlisting[style = lcpp]{construct_huffman_tree.cc}
\end{solution}

\begin{question}
求 $n$ 个数和输入数的海明距离最小的 $k$ 个。
\end{question}
\begin{solution}
关键点:
\begin{compactitem}
    \item 海明距离定义;
    \item 快速求解海明距离;
    \item 取 TopK 时堆的运用。
\end{compactitem}
\lstinputlisting[style = lcpp]{hamming_distance.cc}
\end{solution}

\begin{question}
一个人走楼梯,可以一次走一阶、二阶、三阶,$n$ 阶楼梯有多少种走法。
\end{question}
\begin{solution}
关键点:
\begin{compactitem}
    \item 递推:$S(n) = S(n - 1) + S(n - 2) + S(n - 3)$。
\end{compactitem}
\lstinputlisting[style = lcpp]{steps.cc}
\end{solution}

\begin{question}
给定一个长度为 $n$ 的无序数组,求其连续子序列的最大和。
\end{question}
\begin{solution}
参考 \ref{q:stocks} 的解法。
\end{solution}

\begin{question}
在一个长度为 $n$ 的无序数组中,找到其中第 $k$ 大的元素($k < n$)。要求时间复杂度为 $\Theta(n)$
\end{question}
\begin{solution}
关键点:
\end{solution}

\begin{question}
现有不同时为空的有序数组 A 和 B(非降序),其长度分别为 m 和 n。记 A 和 B 的中位数为 x,要求在 $\Theta(\log(m + n))$ 的时间复杂度内找到 x。
\end{question}
\begin{solution}
关键点:
\begin{compactitem}
\item 理解二分搜索的本质是「在一个有序可随机方位的序列中,寻找某种满足跟『元素顺序』相关条件的元素」;
\item 创造条件,使用二分搜索。
\end{compactitem}
\lstinputlisting[style = lcpp, escapechar = @]{find_median_in_sorted_arrays.cc}
\end{solution}

\begin{question}
给你一棵二叉树,要求填充各节点的 \lstinline[style = lcpp]|next| 指针,以指向节点的下一个兄弟节点;若不存在这样的节点,则指向 \lstinline[style = lcpp]|nullptr|。
\end{question}
\begin{solution}
关键点:
\begin{compactitem}
    \item 层序遍历。
\end{compactitem}
\lstinputlisting[style = lcpp]{fill_next_for_bitree.cc}
\end{solution}


\begin{question}
给你一棵二叉树,要求顺序输出从右往左看能看到的数字。
\end{question}
\begin{solution}
关键点:
\begin{compactitem}
    \item 层序遍历。
\end{compactitem}
\lstinputlisting[style = lcpp]{output_rightmost_node_value.cc}
\end{solution}

\begin{question}
给你一个长度为 \lstinline[style = lcpp]|n| 的向量,实现随机抽样函数 \lstinline[style = lcpp]|random_sample|,从中随机抽样 \lstinline[style = lcpp]|k| 个元素。
\end{question}
\begin{solution}
关键点:
\begin{compactitem}
    \item 入参检查;
    \item 类似线性开放地址法解决 Hash 表冲突的应用;
    \item 线性复杂度。
\end{compactitem}
\lstinputlisting[style = lcpp, escapechar = @]{random_sample.cc}
\end{solution}

\begin{question}
向量 \lstinline[style = lcpp]|std::vector<int>| 是随机变量 $X$ 的所有可能取值,其对应的概率由向量 \lstinline[style = lcpp]|std::vector<double>| 给出。要求实现一个获取随机变量值的函数,使得时间复杂度不超过随机变量可能取值数量的对数复杂度。
\end{question}
\begin{solution}
关键点:
\begin{compactitem}
  \item 构造二分查找的使用场景。
\end{compactitem}
\lstinputlisting[style = lcpp, escapechar = @]{random_choice.cc}
\end{solution}

\begin{question}
向量 \lstinline[style = lcpp]|std::vector<int>| 是随机变量 $X$ 的所有可能取值,其对应的概率由向量 \lstinline[style = lcpp]|std::vector<double>| 给出。要求实现一个获取随机变量值的函数,使得时间复杂度不超过常数复杂度。
\end{question}
\begin{solution}
\lstinputlisting[style = lcpp, escapechar = @]{random_choice_alias_method.cc}
\end{solution}

\section{偏工程、语言问题}
\begin{question}
\label{q:singleton}
实现一个单例模式。
\end{question}
\begin{solution}
关键点:
\begin{compactitem}
    \item 单例的构造,要以 \lstinline[style=lcpp]|std::call_once| 或者 \textsf{pthread} 相关工具保证;
    \item 不能用 double check。
\end{compactitem}
C++ 11 版本:
\lstinputlisting[style = lcpp]{singleton.cc}

\textsf{pthread} 版本:
\lstinputlisting[style = lcpp]{singleton_pthread.cc}
\end{solution}

\begin{question}
C++ 中如何实现一个不可被继承的类?如何使用这种类的对象?
\end{question}
\begin{solution}
关键点:
\begin{compactitem}
    \item 实例化子类需要调用父类构造函数,控制父类构造函数的访问权限为 \lstinline[style=lcpp]|private| 即可;
    \item 也可以用 C++11 的关键字 \lstinline[style=lcpp]|final|;
    \item 类似 \ref{q:singleton} 中的做法。
\end{compactitem}
\end{solution}
\end{document}
